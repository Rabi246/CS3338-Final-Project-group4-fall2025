\documentclass[12pt]{article}

% Packages
\usepackage[margin=1in]{geometry}
\usepackage{graphicx}
\usepackage{longtable}
\usepackage{array}
\usepackage{hyperref}
\usepackage{setspace}
\setcounter{secnumdepth}{4}


\hypersetup{
    colorlinks=true,
    linkcolor=blue,
    urlcolor=blue
}

\begin{document}

%========================
% COVER PAGE
%========================
\begin{titlepage}
    \centering
    \vspace*{2cm}
    {\LARGE \textbf{Software Design Document (SDD)}\par}
    \vspace{1.5cm}
    {\large Project: \textbf{Lab Activity 14}\par}
    \vspace{0.5cm}
    {\large Team Name: \textbf{Tommy W., Nadia H., Damarion M.P, Samantha V.M, Stephany C.P}\par}
    \vspace{0.5cm}
    {\large Group Number: \textbf{4}\par}
    \vspace{0.5cm}
    {\large Version: \textbf{4.0}\par}
    \vfill
    {\large Date: \textbf{December 5, 2025}\par}
    \vspace{2cm}
\end{titlepage}

%========================
% TABLE OF CONTENTS
%========================
\tableofcontents
\newpage

%========================
% VERSION DESCRIPTION
%========================
\section*{Version Description}
\addcontentsline{toc}{section}{Version Description}

\begin{longtable}{|>{\centering\arraybackslash}m{3cm}|
                      >{\arraybackslash}m{7cm}|
                      >{\centering\arraybackslash}m{3cm}|}
    \hline
    \textbf{Version} & \textbf{Description} & \textbf{Date} \\
    \hline
    1.0 &  Snapshot 1. Created system overview and basic architecture. & December 5, 2025 \\
    \hline
    2.0 &  Snapshot 2. Expanded architecture; added AWS + Box components. & December 5, 2025 \\
    \hline
   3.0 & Snapshot 3. Added workflow details and error-handling descriptions. & December 8, 2025 \\
    \hline
   4.0 & Snapshot 4. Final revisions; added UI section, glossary updates, formatting fixes.& December 10, 2025 \\
    \hline
\end{longtable}

\newpage

%========================
% 1. INTRODUCTION
%========================
\section{Introduction}

\subsection{Purpose of the Document}
The purpose of this document is to provide a detail description of the Design details that will be used for Box Discovery Bates Namer. This sofware is a transition from a desktop-based system to a cloud based one. This Sofware Design Document(SSD)  covers the system architecture, user interface and intended audience for the application.

\subsection{Intended Audience}
This document is intended for:
\begin{itemize}
    \item The development team implementing the system.
    \item The course instructor and teaching assistants reviewing the design.
    \item Future developers who may extend or maintain the system.
   \item Various Stakeholders
\end{itemize}

\subsection{Overview of the System}
The \textbf{Box Discovery Bates Namer} is a cloud based solution designed to automate the prossessing the legal discovery PDF files. The applications extrats Bates numbers, Validates their sequence and renames the files accordingly and are stored in Box.com.
\newline
\newline
The System leverages a \textbf{serverless architecture} using \textbf{AWS Lambda functions} for scalability and efficiency. Upon a file upload event to \textbf{Box.com} , the application processes the files in memory to preserve data  confidentiality and complies with data privacy requirements.
\subsubsection{Key Features}

\subparagraph{Integration with Box.com}
\begin{itemize}
    \item Triggered by an HTTPS API Gateway call when files are uploaded to Box.com.
    \item Retrieves file details, including file ID, folder ID, and file name, from the event payload.
\end{itemize}

\subparagraph{Custom Payload Creation}
\begin{itemize}
    \item An initial AWS Lambda function captures the file metadata and creates a custom payload,
          which is passed to the next Lambda function for processing.
\end{itemize}

This system provides a robust and efficient solution for legal professionals by automating discovery file management, enhancing accuracy, and streamlining the workflow.

\newpage

%========================
% 2. SYSTEM ARCHITECTURE
%========================
\section{System Architecture}

\subsection{Overall Workflow of the System}

The Box Discovery Bates Namer follows an automated, event-driven workflow that 
integrates Box.com with AWS Lambda to process, validate, and rename legal discovery
PDF files. The workflow begins when the user uploads a discovery folder into Box.com
and ends when files are renamed, validated, organized, or flagged for errors. The 
following summarizes the complete workflow as described.

\begin{enumerate}
    \item \textbf{User Uploads Discovery Files to Box.com} \\
    A legal professional uploads a folder using the naming format 
    \texttt{PDCase\#\_Disc\#} (e.g., \texttt{PD251234\_02}). Each PDF uploaded 
    triggers the Box Custom Skill, which sends an event to the system.

    \item \textbf{Box Custom Skill Triggers AWS Processing Pipeline} \\
    Box issues a webhook request to an HTTPS API Gateway endpoint, which invokes
    the first Lambda function.

    \item \textbf{BoxInputFunction Captures File Metadata} \\
    The initial Lambda function validates the webhook and extracts file metadata
    including:
    \begin{itemize}
        \item \texttt{file\_id}
        \item \texttt{file\_name}
        \item \texttt{folder\_id}
        \item \texttt{access\_token}
        \item \texttt{user\_id}
    \end{itemize}
    A custom payload is constructed and forwarded to the next Lambda function.
\end{enumerate}
This automated workflow eliminates manual processing, ensures file consistency, 
and provides reliable error reporting for legal professionals.

\subsection{System Components}

The Box Discovery Bates Namer system is built on a serverless architecture using AWS
Lambda and Box.com. The major components are:

\subsubsection{AWS Lambda Functions}
Event-driven functions that perform the core processing:
\begin{itemize}
    \item \textbf{BoxInputFunction} – Validates the webhook and extracts file metadata.
    \item \textbf{BoxFolderGetter} – Retrieves the folder name to identify the PD Case and Disc Number.
    \item \textbf{DiscoveryBatesNamer} – Downloads the PDF, extracts and validates Bates numbers, and generates the new filename.
    \item \textbf{BoxFileUpdater} – Renames and moves the processed file into the correct Box folder.
    \item \textbf{BoxErrorNotification} – Sends user error notifications via email.
\end{itemize}

\subsubsection{Box API}
Handles file storage operations, including:
\begin{itemize}
    \item Downloading files for processing,
    \item Renaming files,
    \item Moving files to the appropriate destination folder.
\end{itemize}

\subsubsection{AWS API Gateway}
Provides the HTTPS endpoint that receives webhook events from Box and triggers the
Lambda pipeline.

\subsubsection{AWS SES}
Sends email notifications to users when a file cannot be processed.

\subsubsection{AWS CloudWatch}
Logs all Lambda activity for debugging and monitoring.

\subsection{Deployment Architecture}

The Box Discovery Bates Namer is deployed entirely using a serverless cloud 
architecture. No traditional servers, containers, or databases are required. All 
processing occurs through AWS-managed services that scale automatically.

\begin{itemize}
    \item \textbf{AWS Lambda} hosts all processing logic for metadata extraction, 
    Bates number parsing, file renaming, and error notifications. Each Lambda 
    function runs independently and is invoked only when needed, reducing cost and 
    improving scalability.

    \item \textbf{AWS API Gateway} provides the public HTTPS endpoint that receives 
    webhook events from Box.com and triggers the first Lambda function.

    \item \textbf{AWS Secrets Manager} stores Box API credentials securely and 
    provides them to Lambda functions at runtime.

    \item \textbf{Box Custom Skill} triggers the pipeline whenever a PDF is uploaded 
    into the designated folder in Box.com.

    \item \textbf{AWS CloudWatch} records execution logs, performance metrics, 
    and error reports for all Lambda functions.
\end{itemize}

Because the system is fully serverless, no container orchestration, local servers, or 
manual deployment environments are required.

\newpage

%========================
% 3. USER INTERFACE
%========================
\section{User Interface}

\subsection{Overview}
The Box Discovery Bates Namer does not require a custom user interface. All user
interaction occurs through the existing Box.com platform. Users upload discovery
folders and PDF files directly into a designated skill-enabled folder in Box, and the
system automatically processes each file through AWS Lambda without requiring further
user input.

\subsection{How Users Interact With the System}
\begin{itemize}
    \item Users log into Box.com and navigate to the skill-enabled “Skills Applied” folder.
    \item The user uploads a folder named using the required format (e.g., \texttt{PD251234\_02}).
    \item As files are uploaded, the system automatically extracts Bates numbers, renames files, and moves them to the appropriate destination folder.
    \item If errors occur (such as missing Bates stamps or incorrect folder naming), the user receives an email notification with details and a link to the affected file.
\end{itemize}

\subsection{User Interface Screens}
The user experience consists of standard Box.com screens, including:
\begin{itemize}
    \item The Box folder view where discovery files are uploaded.
    \item The upload interface used to add PDF files to the designated folder.
\end{itemize}

All processing occurs automatically in the background; therefore, no additional UI elements are required.

\subsection{Data Storage and Management}

The system does not use a traditional relational database. Instead, it relies on 
cloud-managed storage services to maintain the information required for processing
and auditing:

\begin{itemize}
    \item \textbf{Box.com} acts as the primary storage system for discovery files. 
    It stores the original PDFs, renamed documents, and the folder structure used 
    to organize files by PD Case Number and Disc Number.

    \item \textbf{AWS Secrets Manager} stores sensitive credentials such as Box API 
    keys and authentication tokens. These values are securely retrieved by Lambda 
    functions at runtime.

    \item \textbf{AWS CloudWatch Logs} serves as the operational record-keeping 
    system, storing processing logs, error reports, execution traces, and audit 
    information for each file that moves through the pipeline.
\end{itemize}

Together, these services fulfill the system’s storage, security, and traceability needs
without requiring a standalone database engine.

\newpage

%========================
% 4. GLOSSARY
%========================
\section{Glossary}

\begin{longtable}{|>{\centering\arraybackslash}m{3cm}|
                      >{\arraybackslash}m{9cm}|}
    \hline
    \textbf{Acronym} & \textbf{Definition} \\
    \hline
    UI & User Interface \\
    \hline
    API & Application Programming Interface \\
    \hline
    DB & Database \\
    \hline
    SRS & Software Requirements Specification \\
    \hline
    SDD & Software Design Document \\
    \hline
\end{longtable}
\begin{longtable}{|>{\centering\arraybackslash}m{3cm}|
                      >{\arraybackslash}m{9cm}|}
    \hline
    \textbf{Key Deffinitions} & \textbf{Definition} \\
    \hline
    Bates Number &Commonly used in the discovery phase of legal cases, bates numbers are a sequential identifier assigned to each page of a legal document for tracking and referencing  \\
    \hline
    Box.com &  A cloud-based file storage platform used for document organization and collaboration  \\
    \hline
    AWS Lambda & A serverless computing service provided by Amazon Web Services that executes code in response to events, eliminating the need to manage and maintain servers  \\
    \hline
\end{longtable}

\newpage

%========================
% 5. REFERENCES
%========================
\section{References}

\begin{thebibliography}{9}
\bibitem{source}
Santa Barbara Public Defender’s Office. 
\textit{Software Design Document for Box Discovery Bates Namer Integration into Cloud Environment}. 
Version 1.1.2, April 10, 2025. https://ascent.cysun.org/project/project/view/221
\end{thebibliography}


\end{document}

