\documentclass[12pt]{article}

% Packages
\usepackage[margin=1in]{geometry}
\usepackage{graphicx}
\usepackage{longtable}
\usepackage{array}
\usepackage{hyperref}
\usepackage{setspace}

\hypersetup{
    colorlinks=true,
    linkcolor=blue,
    urlcolor=blue
}

\begin{document}

%========================
% COVER PAGE
%========================
\begin{titlepage}
    \centering
    \vspace*{2cm}
    {\LARGE \textbf{Software Design Document (SDD)}\par}
    \vspace{1.5cm}
    {\large Project: \textbf{Lab Activity 14}\par}
    \vspace{0.5cm}
    {\large Team Name: \textbf{Tommy W., Nadia H., Damarion M.P, Samantha V.M, Stephany C.P}\par}
    \vspace{0.5cm}
    {\large Group Number: \textbf{4}\par}
    \vspace{0.5cm}
    {\large Version: \textbf{1.0}\par}
    \vfill
    {\large Date: \textbf{December 5, 2025}\par}
    \vspace{2cm}
\end{titlepage}

%========================
% TABLE OF CONTENTS
%========================
\tableofcontents
\newpage

%========================
% VERSION DESCRIPTION
%========================
\section*{Version Description}
\addcontentsline{toc}{section}{Version Description}

\begin{longtable}{|>{\centering\arraybackslash}m{3cm}|
                      >{\arraybackslash}m{7cm}|
                      >{\centering\arraybackslash}m{3cm}|}
    \hline
    \textbf{Version} & \textbf{Description} & \textbf{Date} \\
    \hline
    1.0 & Initial SDD for Snapshot 1. Includes system overview, architecture, basic UI description, and glossary. & December 5, 2025 \\
    \hline
\end{longtable}

\newpage

%========================
% 1. INTRODUCTION
%========================
\section{Introduction}

\subsection{Purpose of the Document}
This Software Design Document (SDD) describes the overall design of the \textit{Lab Activity 14} system. It explains the system architecture, main components, data flow, and user interface design based on the requirements defined in the SRS.

\subsection{Intended Audience}
This document is intended for:
\begin{itemize}
    \item The development team implementing the system.
    \item The course instructor and teaching assistants reviewing the design.
    \item Future developers who may extend or maintain the system.
\end{itemize}

\subsection{Overview of the System}
The \textbf{Lab Activity 14} system is a structured academic project demonstrating how software documentation is built using LaTeX. It includes a Software Requirements Specification (SRS) and a Software Design Document (SDD) that outline the system’s structure, workflow, interface, and key components.

\newpage

%========================
% 2. SYSTEM ARCHITECTURE
%========================
\section{System Architecture}

\subsection{Overall Workflow of the System}
Example workflow:
\begin{enumerate}
    \item User accesses the application interface.
    \item User submits or interacts with project data.
    \item The back end processes the input and stores information in the database.
    \item Results or responses are shown through the user interface.
\end{enumerate}

\subsection{System Components}
\begin{itemize}
    \item \textbf{Front End:} Provides the user interface using simple UI layouts for interaction.
    \item \textbf{Back End:} Handles logic, data processing, and workflow control.
    \item \textbf{Database:} Stores structured project or user data.
    \item \textbf{External Systems (Optional):} Could include APIs (e.g., authentication or file services).
\end{itemize}

\subsection{Deployment and Container Architecture}

This system is deployed using Docker and Docker Compose. Each main part of the application runs inside its own container so that the environment is consistent across different machines.

\begin{itemize}
    \item \textbf{Frontend Container} -- Runs the React-based user interface. This container serves the web pages that the user interacts with in the browser. It is exposed on port \texttt{3000}.
    \item \textbf{Backend Container} -- Runs the Node.js/Express API that handles file uploads, basic or simulated metadata extraction, and communication with the database. It is exposed on port \texttt{5000}.
    \item \textbf{Database Container} -- Runs PostgreSQL and stores the structured case metadata. This container is not accessed directly by the user; it is only used by the backend container.
\end{itemize}

All of these containers are defined in the \texttt{docker-compose.yml} file at the root of the repository. Docker Compose automatically creates a private network so that the containers can communicate with each other.

The backend container connects to the database using the hostname \texttt{db} and port \texttt{5432}. Configuration values such as database host, port, user, and password are stored in the \texttt{backend/.env} file and loaded at runtime. This keeps configuration separate from the source code and makes the system easier to run in different environments.

To start all containers together in a development environment, the following command is used from the project root:

\begin{verbatim}
docker-compose up --build
\end{verbatim}

To stop and remove the containers, the following command is used:

\begin{verbatim}
docker-compose down
\end{verbatim}

\newpage

%========================
% 3. USER INTERFACE
%========================
\section{User Interface}

\subsection{How to Use the System}
\begin{itemize}
    \item Users access the system through a webpage or interface.
    \item Navigation is provided to move between sections or features.
    \item Forms allow user input such as uploading data or selecting features.
\end{itemize}

\subsection{Database Design and Explanation}
The database consists of organized tables for storing structured information. Tables may include:
\begin{itemize}
    \item User accounts
    \item Documents or records
    \item Metadata related to project workflows
\end{itemize}

\subsection{Screenshots (Optional)}
You may insert screenshots here by uploading files to Overleaf and including them using:
\begin{verbatim}
\includegraphics[width=0.8\textwidth]{filename.png}
\end{verbatim}

\newpage

%========================
% 4. GLOSSARY
%========================
\section{Glossary}

\begin{longtable}{|>{\centering\arraybackslash}m{3cm}|
                      >{\arraybackslash}m{9cm}|}
    \hline
    \textbf{Acronym} & \textbf{Definition} \\
    \hline
    UI & User Interface \\
    \hline
    API & Application Programming Interface \\
    \hline
    DB & Database \\
    \hline
    SRS & Software Requirements Specification \\
    \hline
    SDD & Software Design Document \\
    \hline
\end{longtable}

\newpage

%========================
% 5. REFERENCES
%========================
\section{References}

\begin{itemize}
    \item Course materials and lecture notes.
    \item Overleaf Documentation: \url{https://www.overleaf.com}
\end{itemize}

\end{document}

